\documentclass{article}
\usepackage[utf8]{inputenc}
\usepackage{amsmath}
\usepackage{amsthm}
\usepackage{amssymb}
\usepackage{booktabs}
\usepackage[margin=1in]{geometry}
\usepackage{url}

\newtheorem*{theorem}{Theorem}

\title{Proving that the Boolean algebra forms a vector space}
\date{September 27, 2016}
\author{Bernardo Meurer}
\begin{document}
\maketitle
\newpage
\section{Defining the Boolean algebra}
\label{sec:Defining the Boolean algebra}

We define a Boolean algebra as a set of $B$ elements $a,b,\dots$ which satisfies
the following axioms\cite{booleanalgebra}:

\begin{enumerate}
	\item $B$ has two binary operators $\wedge$ or $\cdot$ (logical AND) and
	      $\vee$ or $+$ (logical OR)
	\item Idempotence
	      \begin{itemize}
	      	\item $a\wedge a = a \vee a = a$
	      \end{itemize}
	\item Commutative law
	      \begin{itemize}
	      	\item $a \wedge b = b \wedge a$
	      	\item $a \vee b = b \vee a$
	      \end{itemize}
	\item Associative law
	      \begin{itemize}
	      	\item $a \wedge (b \wedge c) = (a \wedge b) \wedge c$
	      	\item $a \vee (b \vee c) = (a \vee b) \vee c$
	      \end{itemize}
	\item Absorption law
	      \begin{itemize}
	      	\item $a \wedge (a \vee b) = a \vee (a \wedge b) = a$
	      \end{itemize}
	\item Mutual distributiveness
	      \begin{itemize}
	      	\item $a \wedge (b \vee c) = (a \wedge b) \vee (a \wedge c)$
	      	\item $a \vee (b \wedge c) = (a \vee b) \wedge (a \vee c)$
	      \end{itemize}
	\item $B$ contains universal bounds $\emptyset$ (empty set) and $I$
	      (universal set)
	      \begin{itemize}
	      	\item $\emptyset \wedge a = \emptyset$
	      	\item $\emptyset \vee a = a$
	      	\item $I \wedge a = a$
	      	\item $I \vee a = I$
	      \end{itemize}
	\item $B$ has a unary operator $a\rightarrow a'$ such that
	      \begin{itemize}
	      	\item $a \wedge a' = \emptyset$
	      	\item $a \vee a' = I$
	      \end{itemize}
\end{enumerate}
If the truth values $a, b$ are interpreted as integers $0, 1$ our operators can
be expressed with ordinary arithmetic, or by minimum/maximum functions
\cite{algebrainterp}:

\begin{enumerate}
	\item $a \wedge b = a \times b = \min(a,b)$
	\item $a \vee b = a + b - (a \times b) = \max(a,b)$
	\item $\neg a\text{ or }\bar{a} = 1 - a$
\end{enumerate}
We may also express $a \wedge b$, $a \vee b$, and $\neg a$ with a truth table

\begin{table}
	\parbox{.45\linewidth}{
		\centering
		\begin{tabular}{@{}c c c c @{}}
			\toprule
			$a$ & $b$ & $a \wedge b$ & $a \vee b$ \\ \toprule
			0   & 0   & 0            & 0          \\ \midrule
			0   & 1   & 0            & 1          \\ \midrule
			1   & 0   & 0            & 1          \\ \midrule
			1   & 1   & 1            & 1          \\ \bottomrule
		\end{tabular}
		\caption{Truth table for binary operators}
	}
	\hfill
	\parbox{.45\linewidth}{
		\centering
		\begin{tabular}{@{} c c @{}}
			\toprule
			$a$ & $\neg a$ \\ \toprule
			0   & 1        \\ \midrule
			1   & 0        \\ \bottomrule
		\end{tabular}
		\caption{Truth table for unary operator }
	}
\end{table}

\section{Defining a field}
\label{sec:Defining a field}

We define a field as a triple $(F,+,\cdot)$ where $F$ is a set, and $+$, $\cdot$
are binary operators that act on $F$, called addition and multiplication
respectively, satisfying the following axioms\cite{field}:
\begin{enumerate}
	\item Addition ($+$) is an associative operation on $F$
	      \begin{itemize}
	      	\item $\forall f,g,h \in F : f+(g+h) = (f+g)+h$
	      \end{itemize}
	\item There is an identity element for addition
	      \begin{itemize}
	      	\item $\forall f \in F: f + \nu = f$
	      	\item The identity $\nu$ is unique and we will denote it by   $0$
	      \end{itemize}
	\item Every element $x$ of $F$ is invertible for $+$
	      \begin{itemize}
	      	\item The additive inverse of $x$ is unique, and will be denoted by $-x$
	      \end{itemize}
	\item Multiplication ($\cdot$) is a commutative operation on $F$
	      \begin{itemize}
	      	\item $\forall f,g \in F : f \cdot g = g \cdot f$
	      \end{itemize}
	\item There is an identity element for multiplication
	      \begin{itemize}
	      	\item $\forall f \in F: f \cdot \upsilon = f$
	      	\item The identity $\upsilon$ is unique and we will denote it by $1$
	      \end{itemize}
	\item Every element $x$ of $F$ except $0$ is invertible for $\cdot$
	      \begin{itemize}
	      	\item The multiplicative inverse of $x$ is unique, we will denote it by
	      	      $x^{-1}$
	      	\item We do not assume $0$ to be neither invertible nor non-invertible
	      \end{itemize}
	\item Multiplication is distributive in regards to addition
	      \begin{itemize}
	      	\item $\forall x,y,z \in F : x \cdot (y + z) = (x \cdot y) + (x \cdot z)$
	      \end{itemize}
	\item The identities for addition and multiplication are distinct
	      \begin{itemize}
	      	\item $0 \neq 1$
	      \end{itemize}
\end{enumerate}

One might note that the commutativity of addition is not listed as an axiom,
this is due to the fact that said property can be obtained from the other axioms

\begin{theorem}[Commutativity of addition]

	Let $F$ be any field, then $+$ is a commutative operation on $F$.
	$$\forall f,g \in F :  f + g = g + f$$
\end{theorem}
\begin{proof}\cite{addcomm}

	Let $x, y$ be elements of $F$, from axiom 4.\ we have
	$$(1+x)\cdot(1+y)=(1+y)\cdot(1+x)$$
	Using axiom 7
	$$((1+x)\cdot 1)+((1+x)\cdot y) = ((1+y)\cdot 1)+((1+y)\cdot x)$$
	Axiom 5 gives us that $1$ is the multiplicative identity
	$$(1+x)+((1+x)\cdot y) = (1+y)+((1+y)\cdot x)$$
	Using axiom 1
	$$1+(x+((1+x)\cdot y)) = 1+(y+((1+y)\cdot x))$$
	By means of axiom 3 we have the law of cancellation which yields
	$$x+((1+x)\cdot y) = y+((1+y)\cdot x)$$
	Using axiom 7
	$$x+((1\cdot y)+(x\cdot y)) = y+((1\cdot x)+(y\cdot x))$$
	With axioms 1, 4, and 5
	$$x+y+(x\cdot y) = y+x+(x\cdot y)$$
	Finally by axiom 3
	$$x+y=y+x$$
\end{proof}
\section{Boolean algebra as a field}
\label{sec:Boolean algebra as a field}

If we follow the previously stated definition of a field $F$, and compare it to
the properties imposed on the set $B$, which described our algebra, it will
become clear that $B$ is itself a field.

\begin{table}[h]
	\centering
	\begin{tabular}{@{}lll@{}}
		\toprule
		Property                                                    & Field      & Algebra               \\ \toprule
		Addition is associative                                     & Yes        & Yes                   \\ \midrule
		Addition has an identity element                            & 0          & 0                     \\ \midrule
		Every element $x$ is invertible for $+$                     & $-x$       & $\neg x$ or $\bar{x}$ \\ \midrule
		Multiplication is commutative                               & Yes        & Yes                   \\ \midrule
		Multiplication has an identity element                      & 1          & 1                     \\ \midrule
		Every element $x$ of $F$ except 0 is invertible for $\cdot$ & $x^{-1}$   & $\text{?}$            \\ \midrule
		Multiplication is distributive in regards to addition       & Yes        & Yes                   \\ \midrule
		The identities for addition and multiplication are distinct & $0 \neq 1$ & $0 \neq 1$            \\ \bottomrule
	\end{tabular}
	\caption{Comparing $F$ to $B$}
\label{algebrafieldproof}
\end{table}

\section{Fields as vector spaces}
\label{sec:Fields as vector spaces}

barfoo

\section{Boolean algebra as a vector space}
\label{sec:Boolean algebra as a vector space}

foobar

\newpage

\bibliographystyle{unsrt}
\bibliography{boolean-vector-space}

\end{document}
