\documentclass{article}
\usepackage[utf8]{inputenc}
\usepackage{amsmath}
\usepackage{amssymb}
\usepackage{booktabs}
\title{Proving that the Boolean Algebra forms a vector space}
\date{September 27, 2016}
\author{Bernardo Meurer}
\begin{document}
\maketitle
\newpage
\section{Defining the Boolean algebra}
\label{sec:Defining the Boolean algebra}

We define a Boolean algebra as a set of $B$ elements $a,b,\dots$ with the
following properties:

\begin{enumerate}
  \item $B$ has two binary operators $\wedge$ or $\cdot$ (logical AND) and
        $\vee$ or $+$ (logical OR)
  \item Idempotence
  \begin{itemize}
    \item $a\wedge a = a \vee a = a$
  \end{itemize}
  \item Commutative law
  \begin{itemize}
    \item $a \wedge b = b \wedge a$
    \item $a \vee b = b \vee a$
  \end{itemize}
  \item Associative law
  \begin{itemize}
    \item $a \wedge (b \wedge c) = (a \wedge b) \wedge c$
    \item $a \vee (b \vee c) = (a \vee b) \vee c$
  \end{itemize}
  \item Absorption law
  \begin{itemize}
    \item $a \wedge (a \vee b) = a \vee (a \wedge b) = a$
  \end{itemize}
  \item Mutual distributiveness
  \begin{itemize}
    \item $a \wedge (b \vee c) = (a \wedge b) \vee (a \wedge c)$
    \item $a \vee (b \wedge c) = (a \vee b) \wedge (a \vee c)$
  \end{itemize}
  \item $B$ contains universal bounds $\emptyset$ (empty set) and $I$
  (universal set)
  \begin{itemize}
    \item $\emptyset \wedge a = \emptyset$
    \item $\emptyset \vee a = a$
    \item $I \wedge a = a$
    \item $I \vee a = I$
  \end{itemize}
  \item $B$ has a unary operator $a\rightarrow a'$ such that
  \begin{itemize}
    \item $a \wedge a' = \emptyset$
    \item $a \vee a' = I$
  \end{itemize}
\end{enumerate}
If the truth values $a, b$ are interpreted as integers $0, 1$ our operators can
be expressed with ordinary arithmetic, or by minimum/maximum functions:
\begin{enumerate}
  \item $a \wedge b = a \times b = \min(a,b)$
  \item $a \vee b = a + b - (a \times b) = \max(a,b)$
  \item $\neg a\text{ or }\bar{a} = 1 - a$
\end{enumerate}
We may also express $a \wedge b$, $a \vee b$, and $\neg a$ with a truth table

\begin{table}[]
\parbox{.45\linewidth}{
\centering
\begin{tabular}{@{}c c c c @{}}
\toprule
$a$ & $b$ & $a \wedge b$ & $a \vee b$ \\ \toprule
0   & 0   & 0            & 0          \\ \midrule
0   & 1   & 0            & 1          \\ \midrule
1   & 0   & 0            & 1          \\ \midrule
1   & 1   & 1            & 1          \\ \bottomrule
\end{tabular}
\caption{Truth table for binary operators}
}
\hfill
\parbox{.45\linewidth}{
\centering
\begin{tabular}{@{} c c @{}}
\toprule
$a$ & $\neg a$ \\ \toprule
0   & 1        \\ \midrule
1   & 0        \\ \bottomrule
\end{tabular}
\caption{Truth table for unary operator }
}
\end{table}


\section{Defining a field}
\label{sec:Defining a field}

bar

\section{Boolean algebra as a field}
\label{sec:Boolean algebra as a field}

foobar

\section{Fields as vector spaces}
\label{sec:Fields as vector spaces}

barfoo

\section{Boolean algebra as a vector space}
\label{sec:Boolean algebra as a vector space}

foobar

\end{document}
